%\documentclass{beamer}
%-----------------------------
\documentclass{beamer}
\usepackage{graphicx}
\usepackage{beamerthemeshadow}
%\usepackage{multimedia}

%\titlegraphic{\sound[inlinesound, samplingrate=44100,bitspersample=8,channels=1]{\includegraphics[height=.3\textheight]{slides-voice-icon.png}}{Quiet Man.mp3}






%Set the slide theme
%Change to meet your taste
% Madrid, Copenhagen, Berlin, ... works
%\usetheme{Madrid} 
%\usetheme{metropolis}


\usepackage{xecolor}
\usepackage{amsmath}
%\usefonttheme[onlymath]{serif} %Change the math font
%\settextfont{XB Roya}

%---------------------------------------------------------------------------------
% Seetings to force Beamer works with Xepersian and RTL typesetting
%-------------------------------------------------------------------------------
%\raggedleft

% For right to left lists (itemize and enumerate)
\makeatletter
\newcommand{ \RTList}{\raggedleft\rightskip\@totalleftmargin} 
\makeatother
% Correct the bullet for RTL texts
\setbeamertemplate{itemize item}{\scriptsize\raise1.25pt%
 \hbox{\donotcoloroutermaths$\blacktriangleleft$}} 

% To force beamer use numbering in captions
\setbeamertemplate{caption}[numbered]{}% Number float-like environments



%---------------------------------------------------------------------------------
\title{Book: E-Research: Methods, Strategies, and Issues 1st Edition
page 154 to 156}

\subtitle{Slide Build With a Simple Way‌}
\author{Faeze Qollar}
\institute{Payam Nour University}
\date{Winter 2021} 


\begin{document}
%------------------------------------------
% Title page
%------------------------------------------
\begin{frame}
\maketitle
\end{frame}

% To adjust the paragraphs in RTL
\everypar{\rightskip\rightmargin}
%-------------------------------------------------------------------------------
\begin{frame}{p154}
\section{fundamentals}
\subsection{Simple Text}
facilitated by the use of personalized greetings and friendly language and by the appropriate use of humor. 
Risk 
Reducing risk is accomplished in e-surveys by outlining the ways in which privacy and confidentiality will be protected 
by the researcher. The respondent may also feel at risk because of the length of time required to complete the survey, 
and thus, an estimation of this commitment should be provided. Obviously, the researcher should insure that none of the questions 
or text of any introductory materials insult, embarrass, or deni grate respondents.
Trust 
Trust can be established between the e-researcher and the participants by establishing both personal and institutional or organizational credibility. Provision of a hyperlink to the home page of the researcher, 
as well as to that of any sponsors or institu tional affiliation, also serves to establish feelings of trust.
\end{frame}
%----------------------


%---------------------
%-------------------------------------------------------------------------------
\section{ heading}
%-------------------------------------------------------------------------------
\begin{frame}{Continue p154}
Finally, trust is engendered by building on the commonality and the relationship between researcher and respondent. 
Identification of a common interest in the research questions and a common desire to increase professional competence are ways to build trust in many e-research contexts. 
Theories of persuasion developed from marketing researchcan also be useful in devising ways in which survey response rates can be improved.
In the February 2001 issue of Scientific American, Dr. Robert Cialdini distilled his thirty years of marketing research into six factors that influence a person's decision to respond to a request. These are reciprocation, consistency, social validation, liking, authoring, and scarcity. Reciprocation implies that if the requester offers a gift or some other form of inducement, the respondents will feel obliged to reciprocate by doing the task asked of them.
\end{frame}

%-------------------------------------------------------------------------------
\begin{frame}{P155}
Such inducements could be an offer of an online gift certificate, lottery ticket, or some other gift. Cialdini reports results of including a gift of mailing labels in a request for charity giving resulted in nearly doubling of successful solicitations. Consistency works by reminding the potential respondent of some behavior or indication they have given-of having an interest in the survey (such as responding to an initial letter of invitation). To be self-consistent, the respondent then feels more obligated to complete the survey research process. Social validation refers to the subject wanting to be associated with groups of highly regarded persons. An e-rescarcher can use this per suasion rule in follow-up letters or emails by reporting to nonrespondents that a large group of influential and well-educated people (or other flattering adjectives) have already responded and you are anxious to include their response in this group. The fourth persuasive rule is liking. which can be induced by writing personally and includ ing affective comments in any correspondence with potential respondents.  


\end{frame}
%-------------------------------------------------------------------------------
\begin{frame}{P155}

Revealing personal details(e.g., I am very excited about the potential for this research and its impact on ...) or otherwise creating an affable picture of the researcher in the mind of the respondent can create a sense of attractiveness and empathy. Fifth is authority, which can be built from any connection the research has with prestigious research organizations. Finally, scarcity, or a sense that the opportunity to participate is not available to just anyone, can be used by noting how the sample selection was done and how lucky the respondent is to have been chosen to participate. Of course, these rules can easily be overdone and the credibility of the researcher reduced to marketing hype. However, adapting these rules to the context and purpose of your research and noting how your approach to potential respondents adheres to or violates these rules can significantly affect the research results. 

\end{frame}



%-------------------------------------------------------------------------------
\begin{frame}{CREATING EFFECTIVE C-SURVEY ITEMS}
Most of the techniques and tips for creating paper and pencil surveys are directly relevant to the creation of e-surveys. It is beyond the scope of this text to delve deeply into this subject, and the reader is encouraged to review books or articles specifically focused on survey design such as Dillman (2000) or Alreck and Settle (1985). 
However, there are a number of principles related to survey item construction that are especially relevant to e-surveys. These are: 
• Use as few items as possible. Respondents are busy people; they will not spend long periods of time completing your survey. 
• Make sure that every question directly addresses a significant problem. Try to imagine exactly 
how you will use the results of each question. A question that is of only marginal or of indirect use to your research may be the one that is perceived as "one question too many by a potential respondent and could result in noncompletion of the whole survey. 

\end{frame}	

%-------------------------------------------------------------------------------
\begin{frame}{CREATING EFFECTIVE C-SURVEY ITEMS}
• Keep the items shurt. Few people enjoy reading from a screen, therefore questions should be as short as practically possible. 
• Create simple and direct questions. The less the item lends itself to 
divergent inter pretationsthe more reliable the responses will be. 
• Insure that the items are single-faceted. Often novice researchers attempt to reduce the number of questions by combining two questions in a single item. The result of such a mistake is an inability to determine which of the components (if either) the respondent is answering. 
• Insure all items are bias free. The wording of an item can reflect the opinion or bias of the researcher and thus obscure the respondents' true feelings. For example, asking a question such as why male students use the Internet more then females communicates the researcher's bias by presenting an assumption that may or may not be true. 
 
\end{frame}

%-------------------------------------------------------------------------------
\begin{frame}{CREATING EFFECTIVE C-SURVEY ITEMS}
 
• Use plain language. The U.S. Government maintains a useful plain language guidance site at http://www.plainlanguage.gov/ that provides a tutorial, examples, and reference links to the techniques that help authors write for their intended audiences, use conversational language, and create visually appealing layouts. 
• Use appropriate vocabulary and grammar. When creating survey items, try to put yourself in the mind-set of a projected respondent. Use vocabulary that is appropriate to the educational level and experience of your average respondent. Ap propriate grammar relates to construction of items that are as simple as possible, with little opportunity for confusion by the respondent. For example, a poor question might use double negatives (forcing a respondent to say "yes," when they mean "no"). Survey scales should be equally balanced, with an undecided or not applicable response at the end of the list of options so as to distinguish it from a neutral answer. 
\end{frame}


%-------------------------------------------------------------------------------
\begin{frame}{CREATING EFFECTIVE C-SURVEY ITEMS}

• Use meaningful, mutually exclusive descriptions for scales. Response scales should be logically ordered and should account for at least 90 percent of survey items. If possible, avoid providing more than five to seven response choices. Although electronic space is far cheaper than paper, too inany choices increases cognitive load, clutters the screen, and can result in too much vertical or horizontal scrolling. The practice of not defining a list of intermediate scale choices between two extremes is not recommended as reporting results is challenging and respondents are forced to guess the meaning of their choice.  
\end{frame}


----------
\begin{frame}{CREATING AN EFFECTIVE COVER LETTER}
The email or Web-based introduction to your survey is critical to achieving both high completion rates and quality responses. This communication between you and potential respondents must motivate 
the respondents to feel positively inclined to give you their time to complete the survey. It must also assure the respondents that the research is important and worthy and that they are not opening themselves to any form of risk by assisting you. 
Remember that the Net is not only a distribution channel but it is also a community or, more accurately, thousands of communities. To induce strangers to help you, you must understand their cominunity to the extent that you can appeal to their values and cominunicate to them in a way that inspires both cooperation and trust. 

\end{frame}

----------
\begin{frame}{CREATING AN EFFECTIVE COVER LETTER}
Like all effective communication, knowing and talking directly to the intended audience is critical. The values and ways in which you cominunicate within your research community may be very inappropriate for communication with potential respondents. Pilot studies with subjects who closely match the demographics and worldview of the target audience are thus a critical incans of assessing both the format and the content of this important first communication. E-researchers must also explicitly state (some times more than once) that they are neither selling anything, nor are they using the survey as a way to infiltrate the participant's community or wallet. 
Finally, the increasing use, integration, and sophistication of Net-based tools are giving rise to appropriate concerns about privacy by potential participants. Your cover letter (see sample letter that follows) should be 
explicit about the ethical guidelines that define your research and about the steps that you will take to protect and maintain the privacy of your respondents. 
\end{frame}

\end{document}
